\section{The CellRegMap model} 

CellRegMap builds on and extends the structured linear mixed model (StructLMM \cite{moore2019linear}), which has recently been proposed to test for genotype-environment interactions on physiological traits in population cohorts. 
CellRegMap extends this model to test for interactions between genotype and cellular context on gene expression using single-cell RNA-seq as readout. 
The model is not designed for variant discovery but instead designed to identify and characterize genotype-context (G$\times$C) interactions at known expression quantitative trait loci (eQTL).
In Section~\ref{sec:model}, the CellRegMap model is motivated and derived, followed by the introduction of an efficient scheme for parameter inference and statistical testing.

\subsection{Model definition}
\label{sec:model}
% clarify that samples (N) are cells (or pseudocells)
% maybe start directly from structLMM (not LM, LMM)

Commonly used methods for eQTL mapping based on linear or linear mixed models (LMM) relate individual genetic variants to the expression level of a gene of interest, i.e.,
% \begin{equation}\label{eq:LM}
%  \mathbf{y} =  \mathbf{W}\boldsymbol{\alpha} + \mathbf{g}\beta_G + \boldsymbol{\psi},
% \end{equation}

\begin{equation}\label{eq:LMM}
 \mathbf{y} =  \mathbf{W}\boldsymbol{\alpha} + \mathbf{g}\beta_G + \boldsymbol{\psi}. 
\end{equation}

Here, $\mathbf{y}$ denotes a $N \times 1$ vector of the expression level of a gene of interest across $N$ individuals (typically measured using bulk RNA sequencing); $\mathbf{W}$ is the $N \times P$ design matrix of covariates and $\alpha_i, i=1,..,P$ are the corresponding weights (note that in the main text, i.e., Figure 1d, covariates are omitted for brevity); $\mathbf{g}$ is the $N \times 1$ vector of alleles for each individual at the locus to be assessed and $\beta_G$ denotes the corresponding effect size.
% The variable $\mathbf{u}$ denotes a random effect term, which can be used to account for additive confounding factors or relatedness, and f
Finally $\boldsymbol{\psi}$ denotes i.i.d. noise, $\boldsymbol{\psi} \sim \mathcal{N}(\mathbf{0}, \sigma_n^2\mathbf{I})$.
Additional random effect components have been introduced to account for relatedness between individuals~\cite{hoffman2013correcting}, or to adjust for additive confounding sources of variation on gene expression~\cite{fusi2012joint}. 
% \\

\subsubsection*{Review of StructLMM}
StructLMM \cite{moore2019linear} extends the conventional linear association test in Eq.~\eqref{eq:LMM} by including an additional random effect component to account for heterogeneity in effect sizes across individuals due to context-specific genetic effects. 
Briefly, StructLMM can be cast as:

\begin{equation}\label{eq:StructLMM}
 \mathbf{y} =  \mathbf{W}\boldsymbol{\alpha} + \mathbf{g}\beta_G + \mathbf{g} \odot \boldsymbol{\beta_{GxC}} + \mathbf{c} +\boldsymbol{\psi}, 
\end{equation}

where $\mathbf{g} \odot \boldsymbol{\beta_{GxC}}$ accounts for G $\times$ C and $\mathbf{c}$ for additive contributions of environmental variation. 
Unlike in conventional interaction tests, genotype-context interactions due to environmental variation are accounted by introducing an additional genetic effect with per-individual effects, where each individual has its own effect size.
The symbol $\odot$ denotes the Hadamard product and $\boldsymbol{\beta_{GxC}}=[\beta_{GxC_1}, .. ,\beta_{GxC_N}]^T$ is a vector of per-individual effect sizes to account for heterogeneous genetic effects.
Instead of explicitly estimating the GxC effect sizes, these parameters are marginalised under a multivariate normal prior distribution that is defined by an environmental context covariance matrix

\begin{equation}\label{eq:beta_GxE}
    \boldsymbol{\beta_{GxC}} \sim \mathcal{N}(\mathbf{0},\sigma_{GxC}^2 \boldsymbol{\Sigma}).
\end{equation} 

In this case the covariance $\boldsymbol{\Sigma}$ does not encode relatedness as in a classical LMM, but instead accounts for sample covariance due to different environmental states.
The same environmental covariance matrix can also be used to parameterize a second random effect component to account for additive environmental contributions,

\begin{equation}
   \mathbf{c} \sim \mathcal{N}(\mathbf{0},\sigma_c^2 \boldsymbol{\Sigma}). 
\end{equation}

Notably, StructLMM does not account for any additional structure across samples, such as population structure or repeat measurements.
While discrete populations can be accounted for as covariates, more subtle relatedness or repeat structure cannot be effectively encoded using fixed effect covariates. 
Consequently, the model is not suitable for any experimental design where multiple observations are available for the same individual, which is the case in single-cell genetic studies where multiple cells are assayed from each individual. 
Such a repeat structure results in un-calibrated p-values; see also the results from the simulation study to assess calibration (main text Fig. 2). 
% \\

\subsubsection*{CellRegMap}
CellRegMap extends StructLMM to allow for applications to single-cell expression data by introducing an additional random effect component that accounts for relatedness or sample repeat structure. 
First we note that the phenotype $\mathbf{y}$ now represents single-cell resolved expression data, so our samples are expression levels in cells, not individuals. 
This introduces additional structure in the data, as typically multiple cells are sampled from the same individual.
To account for this structure, an additional random effect component $\mathbf{u}$ is included in the model: 

\begin{equation}\label{eq:scStructLMM}
 \mathbf{y} =  \mathbf{W}\boldsymbol{\alpha} + \mathbf{g}\beta_G + \mathbf{g} \odot \boldsymbol{\beta_{GxC}} + \mathbf{c} + \mathbf{u} + \boldsymbol{\psi}. 
\end{equation}

Here, $\mathbf{y}$ has the cardinality of single-cells rather than individuals, and hence the G$\times$C component ($\mathbf{g} \odot \boldsymbol{\beta_{GxC}}$) accounts for interactions with cellular states and contexts (which are well defined at the level of single cells) as well as environmental exposures and stimuli (which can also be individual-level).
The terms $\mathbf{c} \sim \mathcal{N}(\mathbf{0},\sigma_c^2 \boldsymbol{\Sigma})$ and $\boldsymbol{\psi} \sim \mathcal{N}(\mathbf{0},\sigma_n^2\mathbf{I})$ have the same meaning as previously, noting that for these $N \times 1$ vectors $N$ now represents the total number of cells, not the number of unique individuals.
Similarly, $\boldsymbol{\Sigma})$ is $N \times N$ and is again defined in the space of cells.
The symbol $\boldsymbol{\beta_{GxC}}\sim \mathcal{N}(\mathbf{0},\sigma_{GxC}^2 \boldsymbol{\Sigma})$ now represents cell-level effect sizes, which again captures variation in genetic effects across cells.
% to account for heterogeneous genetic effects, which follows a multivariate normal distribution:
% \begin{equation}\label{eq:beta_GxE}
%     \boldsymbol{\beta_{GxE}} \sim \mathcal{N}(\mathbf{0},\sigma_{GxE}^2 \boldsymbol{\Sigma}),
% \end{equation}
% Depending on the functional form of the environmental covariance $\boldsymbol{\Sigma}$, this model can account for different types of G$\times$E, for example, both discrete and continuous cell states and types or donor-level environmental covariates. 
% The same environmental covariance matrix is also used to account for additive environmental effects:
% \begin{equation}
%   \mathbf{e} \sim \mathcal{N}(\mathbf{0},\sigma_e^2 \boldsymbol{\Sigma}). 
% \end{equation}
% Finally, the model includes a background term that accounts for interactions of repeat structure ($\mathbf{R}$) and environments ($\boldsymbol{\Sigma}$):
The additional random effect component $\mathbf{u}$ accounts for relatedness or the repeat structure, which is parameterized as a product kernel between relatedness ($\mathbf{R}$) and the environmental covariance ($\boldsymbol{\Sigma}$):

\begin{equation}
    \mathbf{u} \sim \mathcal{N}(\mathbf{0},\sigma_{rc}^2 \mathbf{R} \odot \boldsymbol{\Sigma}).
\end{equation}

Here, $\mathbf{R}$ denotes the relatedness matrix of individuals expanded to all cells based on the known assignment of cells to individuals and the covariance $\mathbf{\Sigma}$ again denotes the cell-level environmental context.
Notably, this parametrization extends the classical LMM, which would exclusively consider a relatedness component $\mathbf{R}$.
One way to interpret this covariance is to account for polygenic interactions between environmental and relatedness, which has previously been considered to estimate the GxE component of heritability~\cite{heckerman2016linear}.


% Finally, we include a background term ($\mathbf{u}$) to model non-random interactions between the repeat structure and such contexts. 

% where we note that $\mathbf{R}$ is appropriately expanded to reflect the repeated structure of multiple cells derived from the same individuals. 

\subsection{Construction of the cellular context covariance}

Typically, we define $\boldsymbol{\Sigma} = \mathbf{E}\mathbf{E}^T$, and hence the cellular context covariance is a linear function of a matrix of environmental states $\mathbf{E}$.
In practice, we consider as cellular contexts axes of variation in the dataset (for example captured by principal components or MOFA \cite{argelaguet2018multi} factors), appropriately standardized (mean=0, standard deviation=1) and build $\boldsymbol{\Sigma} = \mathbf{E}\mathbf{E}^T$ accordingly.
Depending on the type and structure of cellular contexts, $\boldsymbol{\Sigma}$ can simply separate cells into groups, and appear as a block diagonal or capture continuous transitions (main text Fig. 1c). 
In principle, CellRegMap can also be use in conjunction with other parameterizations of the cell context covariance.

\subsection{Statistical testing}

The main operations in the CellRegMap model, including parameter inference and tests are implemented in its marginalised form. 
Integrating over the random effect components, the marginal likelihood of the model in Eq. \eqref{eq:scStructLMM} follows as:

\begin{equation}
    \label{eq:scStructLMM_marginal}
     \mathbf{y} \sim \mathcal{N} ( \mathbf{W}\boldsymbol{\alpha} + \mathbf{g}\beta_G, \sigma^2_{GxC} \text{diag}(\mathbf{g}) \boldsymbol{\Sigma} \text{diag}(\mathbf{g}) + \sigma_c^2 \boldsymbol{\Sigma} + \sigma_{rc}^2 \mathbf{R} \odot \boldsymbol{\Sigma}+ \sigma_n^2 \mathbf{I} ). 
\end{equation}

While in principle CellRegMap can be used to test for different components, including additive genetics, interactions or the combination of both effects (see~\cite{moore2019linear} for details on StructLMM), we here focus on GxC effects. 
In order to evaluate the significant contribution of GxC effects, we consider a statistical test that compares the following hypotheses, from Eq.\eqref{eq:scStructLMM}:

\begin{equation*}
    H_0: \sigma_{GxC}^2 = 0,
\end{equation*}

\begin{equation*}
    H_1: \sigma_{GxC}^2 > 0.
\end{equation*}

We use Rao's Score test~\cite{rao1948large} to evaluate significance, which allows us to only calculate the MLE\footnote{maximum likelihood estimator} of the parameters under the null hypothesis $H_0$, which follows from~Eq.\eqref{eq:scStructLMM_marginal} as:

\begin{equation}\label{eq:scStructLMM_H0_MVN}
 \mathbf{y}|H_0 \sim \mathcal{N}( \mathbf{W}\boldsymbol{\alpha} + \mathbf{g}\beta_G, \sigma_c^2 \boldsymbol{\Sigma} + \sigma_{rc}^2 \mathbf{R} \odot \boldsymbol{\Sigma}+ \sigma_n^2 \mathbf{I} ). 
\end{equation}
% \\

To test for GxC interactions we adapt the score-based testing scheme employed in StructLMM, which in turn adopts fast LMM testing in LMMs as first proposed in Lippert et al., \cite{lippert2011fast}.
We here review the key steps involved.
\\

First, we define the score-based test statistic $\mathrm{Q}$ as:

\begin{equation}\label{eq:Q}
    \mathrm{Q} = \frac{1}{2}\mathbf{y}^T\mathbf{P}_0 \frac{\partial \mathbf{K}}{\partial \theta}\mathbf{P}_0 \mathbf{y}, 
\end{equation}

where $\mathbf{K}$ denotes the full covariance matrix, i.e., from Eq. \eqref{eq:scStructLMM}:

\begin{equation}\label{eq:full_K_scStructLMM}
    \mathbf{K} = \sigma_{GxC}^2\mathrm{diag}(\mathbf{g})\boldsymbol{\Sigma} \ \mathrm{diag}(\mathbf{g}) +  \sigma_c^2 \boldsymbol{\Sigma} + \sigma_{rc}^2 \mathbf{R}\odot\boldsymbol{\Sigma}+ \sigma_n^2 \mathbf{I},
\end{equation}

and 

\begin{equation}
    \mathbf{P}_0 = \mathbf{K}_0^{-1}-\mathbf{K}_0^{-1}\mathbf{X}(\mathbf{X}^T\mathbf{K}_0^{-1}\mathbf{X})^{-1}\mathbf{X}^T\mathbf{K}_0^{-1}
\end{equation}

is a matrix that projects out the fixed effects \cite{lippert2011fast, lippert2014greater}.
In our case (Eq.\eqref{eq:scStructLMM}), the fixed effects include covariates and the persistent effect of the variant tested: $\mathbf{X} = [\mathbf{W}, \mathbf{g}]$, and $\mathbf{K}_0$ is as in Eq.\eqref{K0}.
Using Eq.\eqref{eq:full_K_scStructLMM} and considering the parameter $\theta = \sigma_{GxC}^2$, we can derive:

% \begin{equation}
%     \frac{\partial \mathbf{K}}{\partial \sigma_{GxE}^2} = \mathrm{diag}(\mathbf{g})\mathbf{E}\mathbf{E}^T\mathrm{diag}(\mathbf{g}).
% \end{equation}

% Next, let us define $\mathbf{K}_1 = \mathrm{diag}(\mathbf{g})\mathbf{E}\mathbf{E}^T\mathrm{diag}(\mathbf{g})$and substituting in eq. \eqref{eq:Q}, we can write:

\begin{equation}
    \frac{\partial \mathbf{K}}{\partial \sigma_{GxC}^2} = \mathrm{diag}(\mathbf{g})\boldsymbol{\Sigma} \ \mathrm{diag}(\mathbf{g}).
\end{equation}

Next, let us define $\mathbf{K}_1 = \mathrm{diag}(\mathbf{g})\boldsymbol{\Sigma} \ \mathrm{diag}(\mathbf{g})$; substituting in Eq.\eqref{eq:Q}, we can write:

\begin{equation}
    \mathrm{Q} = \frac{1}{2}\mathbf{y}^T\mathbf{P}_0 \mathbf{K}_1\mathbf{P}_0 \mathbf{y}. \\
\end{equation}

% \newpage

As before, $H_1: \sigma_{GxC}^2>0$, noting that as a variance parameter, $\sigma_{GxC}^2$ is constrained to take on positive values.
As a result, the score test statistic $\mathrm{Q}$ follows a mixture of $\chi^2$ distributions\footnote{We refer the reader specifically to the supplementary methods from \cite{lippert2014greater} for a proof.}:

\begin{equation}
    \mathrm{Q} \sim \sum_i \lambda_i \chi^2_1, 
\end{equation}

where $\lambda_i$'s are the non-zero eigenvalues of $\frac{1}{2}\mathbf{P}_0^{\frac{T}{2}} \frac{\partial\mathbf{K}}{\partial \theta} \mathbf{P}_0^{\frac{1}{2}}$.\\

It can be shown that for a matrix $\mathbf{A}$ the non-zero eigenvalues of $\mathbf{A}\mathbf{A}^T$ are the same as those of $\mathbf{A}^T\mathbf{A}$ , thus we can re-arrange and compute $\lambda_i$'s as the eigenvalues of:

\begin{equation}
    \frac{1}{2}\frac{\partial\mathbf{K}}{\partial \theta}^{\frac{T}{2}} \mathbf{P}_0 \frac{\partial\mathbf{K}}{\partial \theta}^{\frac{1}{2}}
\end{equation}

instead. 
% \\
To evaluate the significance of the score-best test statistic $\mathrm{Q}$ we use the approach described in Sequence Kernel Association Test (SKAT \cite{wu2011rare}), thereby using the Davies exact method \cite{davies1980algorithm} to compute the corresponding p values, and switching to the modified moment matching approximation method \cite{liu2009new, lee2012optimal, duchesne2010computing} when this fails to converge.

\subsection{Implementation}

To enable efficient parameter inference, we extended the strategy in StructLMM~\cite{moore2019linear}, which builds on the reparametrization of the LMM likelihood proposed in~\cite{lippert2011fast}. 
Briefly, the key is to rewrite the overall covariance of $\mathbf{y}|H_0$ in the form: $\sigma^2_m(\mathbf{M}+\delta\mathbf{I})$, where $\mathbf{M}$ is ideally low rank to enable an efficient singular value decomposition.  
To do so, we introduce a weight parameter $\rho_1$, such that the covariance matrix of $\mathbf{y}$ under the null hypothesis, $\mathbf{K}_0 = \mathrm{Cov}(\mathbf{y} | H_0)$ can be cast as:

\begin{equation}\label{K0}
\begin{split}
    \mathbf{K}_0 = \sigma_c^2 \boldsymbol{\Sigma} + \sigma_{rc}^2 \mathbf{R} \odot \boldsymbol{\Sigma}+ \sigma_n^2 \mathbf{I} =\\
    \sigma_m^2 \ [\rho_1\boldsymbol{\Sigma} + (1-\rho_1) \mathbf{R} \odot \boldsymbol{\Sigma}] + \sigma_n^2 \mathbf{I} =\\ \sigma_m^2 \{ {\mathbf{M}(\rho_1) + \delta_1 \mathbf{I}\}},
\end{split}
\end{equation}

where $\sigma_m^2\rho_1 = \sigma_c^2$,
$\sigma_m^2(1-\rho_1) = \sigma_{rc}^2$,
$\delta_1 = \sigma_n^2/\sigma_m^2$, and $\mathbf{M}(\rho_1) = \rho_1\boldsymbol{\Sigma} + (1-\rho_1) \mathbf{R} \odot \boldsymbol{\Sigma}$.\\

We note that $\mathbf{M}$ only depends on $\rho_1$. 
The parameter $\rho_1$ is optimized using a grid search with values in the range 0 to 1, selecting the value of $\rho_1$ that maximises the likelihood of $\mathbf{y}|H_0$ under the null.
Once $\rho_1$ is fixed, the decomposition of $\rho_1\boldsymbol{\Sigma} + (1-\rho_1) \mathbf{R} \odot \boldsymbol{\Sigma}$ can be found using the observation that the decomposition of the linear combination of two square symmetric matrices, e.g.,:

\begin{equation}
    \mathbf{M} = a\mathbf{A} + b\mathbf{B}
\end{equation}

can be found as a function of the decomposition of the two original matrices, i.e.,:

\begin{equation}
    \mathbf{N} = [\sqrt{a}\mathbf{C} | \sqrt{b}\mathbf{D}]
\end{equation}

such that $\mathbf{M}=\mathbf{N}\mathbf{N}^T, \mathbf{A}=\mathbf{C}\mathbf{C}^T$ and $\mathbf{B}=\mathbf{D}\mathbf{D}^T$. \\

In our case, $\mathbf{A} = \boldsymbol{\Sigma}$ and $\mathbf{B} = \mathbf{R} \odot \boldsymbol{\Sigma}$.
If $\boldsymbol{\Sigma} = \mathbf{E}\mathbf{E}^T$, its decomposition is straight-forward, i.e., $\mathbf{C} = \mathbf{E}$.
The decomposition of the second term ($ \mathbf{R} \odot \boldsymbol{\Sigma}$), assuming $\boldsymbol{\Sigma} = \mathbf{E}\mathbf{E}^T$ and $\mathbf{R} = \mathbf{G}\mathbf{G}^T$ is a bit less trivial.
Assuming that the rank of $\mathbf{R}$ is the number of individuals, that the rank of $\boldsymbol{\Sigma}$ is the number of environments, and that the latter is smaller than the former, we can obtain $\mathbf{D}$ as the decomposition of $\mathbf{R} \odot \boldsymbol{\Sigma}$ by:

\begin{equation}
    \mathbf{R} \odot \boldsymbol{\Sigma} = \mathbf{R} \odot \sum_i[\mathbf{v}_i \lambda_i \mathbf{v}_i^T] = \sum_i[\textcolor{red}{\mathbf{R} \odot (\mathbf{v}_i \lambda_i \mathbf{v}_i^T)}].
\end{equation}

% \newpage

The terms in the sum can be rewritten as:

\begin{equation}\label{eq:Lk}
    \begin{split}
        \textcolor{red}{\mathbf{R} \odot (\mathbf{v}_i \lambda_i \mathbf{v}_i^T)} 
        & = \mathbf{R} \odot (\mathbf{u}_i \mathbf{u}_i^T) \\
        & = \mathbf{R} \odot (\mathrm{diag}(\mathbf{u}_i) \mathbf{e} \mathbf{e}^T \mathrm{diag}(\mathbf{u}_i)) \\
        & = \mathrm{diag}(\mathbf{u}_i)[\mathbf{R} \odot \mathbf{e} \mathbf{e}^T] \mathrm{diag}(\mathbf{u}_i) \\
        & = \mathrm{diag}(\mathbf{u}_i) \mathbf{R} \mathrm{diag}(\mathbf{u}_i) \\
        & = \mathrm{diag}(\mathbf{u}_i) \mathbf{G} \mathbf{G}^T  \mathrm{diag}(\mathbf{u}_i)  \\
        & = \textcolor{red}{\mathbf{Lk}_i \mathbf{Lk}_i^T},
    \end{split}
\end{equation}

where $\mathbf{Lk}_i = \mathrm{diag}(\mathbf{u}_i) \mathbf{G}$; $\lambda_i$'s and $\mathbf{v}_i$'s are eigenvalues and eigenvectors of $\mathbf{E}\mathbf{E}^T$; $\mathbf{u}_i = \sqrt{\lambda_i}\mathbf{v}_i$ and $\mathbf{e}$ is a vector of ones ($\mathbf{e} = [1..1]$). \\

% \newpage

Thus, $\mathbf{M} = \boldsymbol{\Sigma} + \mathbf{R} \odot \boldsymbol{\Sigma}$ can be written as the following sum:

\begin{equation}
    \mathbf{M} = \boldsymbol{\Sigma} + \sum_i \mathbf{Lk}_i \mathbf{Lk}_i^T,  
\end{equation}

from which follows:

\begin{equation}
    \mathbf{N} = [\mathbf{E} \ | \mathbf{Lk}_1 \ | .. \ | \mathbf{Lk}_{ne}],
\end{equation}

where $ne$ is the number of contexts considered (and the rank of $\boldsymbol{\Sigma}$).

\newpage

\subsection{Computational complexity}
From using the implementation described above, it follows that the complexity is O(N) where N is the minimum between the number of cells and the product of number of unique individuals $\times$ the number of cellular contexts. Runtimes were also evaluated empirically, using simulated phenotypes for one gene, 25 SNPs and 20 different contexts (see section \ref{sec:simulations}). We considered 25 and 50 individuals, respectively, and assessed computational runtimes as a function of the number of cells per individual (25, 50, 75, 100, 125, 150) using a 2018 MacBook Pro with 2,3 GHz Quad-Core Intel Core i5 processor (Supp. Fig. 1.1).


% Various methods have been proposed to compute the tail probability of the mixture of 1-dof $\chi^2$ distributions. 
% For example, we can use the Davies method \cite{davies1980algorithm}, the moment‐matching‐based noncentral $\chi^2$ approximation method \cite{liu2009new, lee2012optimal}, or the saddlepoint approximation method \cite{kuonen1999miscellanea}. 
% The Davies methos is the most accurate but can be be computationally expensive.
% The moment-matching approximation is anti-conservative and could lead to inflated type I errors especially for small significance levels.
% SKAT: Davies + Liu
% SKATh: Davies + saddle
% \cite{wu2016efficient}

% \section{Definition of covariance matrix}

% E?

% PCs, MOFA, HVGs?
% transformation?
% normalisation/scaling

% \begin{figure}[h]
% \centering
% \includegraphics[width=15.5cm]{Chapter6/Fig/sc_structlmm_calibration.png}
% \caption[Calibration QQ-plots]{\textbf{Calibration QQ-plots}.\\
% (a) No genetic effect at all.
% (b) No GxE effect (persistent only).}
% \label{fig:sc_structlmm_calibration}
% \end{figure}

% \subsection{Comparison with StructLMM v0}

% Next, we compared our model to the original StructLMM model \cite{moore2019linear}.
% StructLMM is expected to have issues in the presence of extended repeated structure, which it is not equipped to deal with.
% To address this, we simulated various numbers of repeats per donor - mimicking cells -  ranging from 10 to 500.
% Indeed, we observe over-inflation of StructLMM in the presence of many repeats making the model not calibrated (Fig. Xa).
% sc-StructLMM, on the other hand, is nicely calibrated (Fig. Xb).

% \subsection{Comparison with standard interaction test}

% We then performed power analysis when comparing our model with one where the environments are modelled as fixed effects (see Section 2.4.2).

% % \clearpage

% \section{Application to differentiating iPS cells}

% % Next, 
% I applied sc-StructLMM on the data described in \textbf{Chapter 
% % \ref{chapter4}
% 4} (and published in \cite{cuomo2020single}).
% % We applied our model on a recently published dataset.
% Briefly, human iPS cells (\textbf{section 
% % \ref{sec:ipsc}
% 1.2}) from 125 individuals are differentiated from a pluripotent state, to definitive endoderm. 
% % \subsection{Analysis of factors using MOFA}
% I use 
% % principal components (PCs) 
% multi-factor analysis (MOFA \cite{argelaguet2018multi}) to identify factors that
% % 10 MOFA factors estimated from the data \cite{argelaguet2018multi}
% % 500 independent ($r^2<0.2$) highly variable genes (HVGs)
% % to 
% capture various aspects of the variation in gene expression in the dataset, which represent cell states and types. 
% In particular, the MOFA factor1 aligns with the differentiation axis (similar to PC1, see \textbf{section 
% % \ref{sec:endodiff_sources_of_variation}
% 4.3.1}). 
% Factor 3, on the other hand, captures cells in different phases along the cell cycle, as estimated using Seurat \cite{stuart2019comprehensive} (\textbf{Fig. \ref{fig:sc_structlmm_endo_mofa_factors}}). 
% \\

% % \begin{figure}[h]
% % \centering
% % \includegraphics[width=15.5cm]{Chapter6/Fig/MOFA_factors.png}
% % \caption[MOFA factors used as contexts]{\textbf{MOFA factors used as contexts}.\\
% % Evaluation of some of the factors identified using multi-omics factor analysis (MOFA \cite{argelaguet2018multi}).
% % (left) MOFA1 and MOFA2 mostly reflect the developmental stage of our cells, with MOFA1 (x axis) recapitulating a differentiation trajectory from iPSCs (day0) to definitive endoderm (day3).
% % MOFA3 vs MOFA6: cell cycle.}
% % \label{fig:sc_structlmm_endo_mofa_factors}
% % \end{figure}

% First, I used the first 10 factors calculated using MOFA as cellular states in the model (i.e. as columns of $\mathbf{E}$ from eq. \eqref{eq:scStructLMM}).
% For numerical reasons, the factors were quantile normalised, prior to building the covariance matrix (as $\mathbf{E}\mathbf{E}^T$). 
% % \\
% Next, I tested the combined set of eQTL identified at any of the developmental stages defined in \textbf{Chapter 
% % \ref{chapter4}
% 4} (iPSC, mesendo, defendo - for a total of 4,470 gene-SNP pairs) for GxE interactions.
% Single cell expression of each of the eGenes tesed ($\mathbf{y}$) was also quantile-normalised. \\

% To assess the effect of including more cell states in the model, I tested the model using only one factor, the first two factors, the first five, and finally all ten factors (\textbf{Fig. \ref{fig:sc_structlmm_endo_barplots}}).

% \newpage

% % \subsection{Data prep}

% % Both y and E are quantile normalised,..

% \subsection{Permutation strategy for multiple testing correction}

% To correct for multiple testing correction, I used a permutation approach similar to that described in \cite{ongen2016fast}.
% In particular, I ran 100 permutations for each of the eQTL tested, where I permuted the values of the cell states (i.e. the MOFA factors) within each donor, across cells.
% Next, for each set of permutations, I select the minimum (permuted) p value, across all eQTL.
% I then use these 100 permuted p values to estimate a beta distribution and correct the original real p values.

% \subsection{Results}

% We observe an increase in our power to identify `interaction eGenes' (i.e. genes with at least one GxE interaction eQTL) as we increase the number of 
% factors included as contexts 
% % PCs used as environments, with then a plateau at 50 PCs
% (\textbf{Fig. \ref{fig:sc_structlmm_endo_barplots}}). \\

% % \begin{figure}[htbp]
% % \centering
% % \includegraphics[width=14cm]{Chapter6/Fig/sc_structlmm_endodiff_mofa.png}
% % \caption[Results on endodiff data]{\textbf{Results on endodiff data}.\\
% % Barplots showing the number of eGenes compared to the genes tested.
% % Endodiff significant eGenes only (iPSC+mesendo+defendo, FDR < 10\%).
% % Chromosome 22 only (88 genes).}
% % \label{fig:sc_structlmm_endo_barplots}
% % \end{figure}

% % Reassuringly, we could recapitulate most (XX\%) of the dynamic eQTL described in the original study (\textbf{section 
% % % \ref{sec:endodiff_dynamic_eqtl}
% % 4.5}), which were detected using solely PC1, 
% % % and using ASE and a linear model (see eq. \eqref{eq:endodiff_ase_pseudotime}
% % % 4.1).
% % % and a fixed effect linear mixed model (Methods). 
% % However, we show that we have increased power using our model. 
% % Furthermore, the overlap with our interaction eQTL and the dynamic eQTL identified in Cuomo \textit{et al}. decreases, the more PCs we include (Fig. XX).\\



% % As environments, we used
% % the first 10 principal components


% % PCs, MOFA, HVGs?
% % transformation?
% % normalisation/scaling




% % \section{Upstream analysis}

% % One of the key steps in running this method is choosing the environmental factors.
% % We expect most applications to be for scRNA-seq datasets only (and genotypes).
% % This means that the environments need to be estimated from the transcriptomic data, which is also used as phenotype, causing concerns of circularity.


% % A user might have 
% % PCA
% % MOFA (single omic)

% % \section{Downstream analysis} 


