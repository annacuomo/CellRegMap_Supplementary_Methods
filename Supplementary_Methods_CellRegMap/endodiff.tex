\section{Application to endoderm differentiation}\label{sec:endodiff}

We considered single-cell expression data from 125 individuals from \cite{cuomo2020single}.
These data cover \textit{in vitro} differentiation of iPS cells from pluripotent stage (day0) to definitive endoderm (day3).
Compared to the entire dataset considered in the original publication, we discarded two outlying cell sub-populations (n=475 and n=1,212 respectively).

\subsection{Preprocessing of scRNA-seq count data}

Count data were processed as in the primary paper \cite{cuomo2020single}, where counts were normalised using scran \cite{lun2016pooling} and log-transformed ($log_2(x+1)$).
Prior to be inputed into the model as phenotype vectors (i.e., as $\mathbf{y}$), single-cell counts for a given gene were quantile-normalized to better fit the Gaussian distribution assumed by the model.
The log-normalized count data (prior to quantile normalization) for the top 500 highly variable genes is also used as input for MOFA (see details below).

\subsection{Estimation and annotation of MOFA factors}

MOFA \cite{argelaguet2018multi} factors calculated from the expression profiles of the top 500 highly variable genes identified across all cells (using scran’s function “modelGeneVar”), using default parameters. 
Annotation of the leading 10 MOFA factors was performed using gprofiler \cite{raudvere2019g}, considering the absolute loading of individual genes as estimated by MOFA.   
For each MOFA factor, we considered the top 20 enriched pathways to annotate individual factors (adjusted p-values $<$ 0.05; Supplementary Fig. 3.1).

\subsection{Testing for G$\times$C effects}

We considered \emph{cis}-eQTL from \cite{cuomo2020single}. 
In particular, we considered all gene-SNP pairs that were  significant (FDR $<$ 10\%) in one or more of the developmental stages considered in the original paper, i.e., iPSCs, mesendoderm, definitive endoderm. 
In total, we considered 4,470 eQTL pairs (3,240 eGenes).
This approach is typical to GxE interaction testing, where SNPs with at least weak persistent effects (g only) are used as good candidates to display GxE effects. 
\\

Next, we mapped context-specific effects using either 1 or 10 MOFA factors. 
In both cases, all 20 MOFA factors were used to construct the background term, i.e., columns in $\mathbf{E}$ standardized and then used to build $\boldsymbol{\Sigma}=\mathbf{E}\mathbf{E}^T$.
To account for the multiple testing burden, the resulting p-values were corrected at the gene-level using the Bonferroni procedure to control the family-wise error rate, and across genes using the Storey method to control the false-discovery rate (FDR). 
Significant results were reported at FDR $<$ 5\%. 

\subsection{Estimation of single-cell effect sizes}

We estimated both persistent genetic effects ($\beta_G$) and cell-level effect sizes due to G$\times$C ($\boldsymbol{\beta_{GxC}}$; as described in Section \ref{sec:beta_gxe}) for the top significant context-specific eQTL identified (FDR $<$ 5\%), when considering either only one MOFA factor (MOFA 1, capturing differentiation) or the first 10 MOFA factors as cell contexts.
% each of 1, 2, 5, 10 and 20 MOFA factors as contexts.
% For each eQTL, the magnitude of GxC driven effects was calculated as the difference between the 0.9 and 0.1 quantiles from the cell-level effect size estimates.

% \subsection{Clustering of single-cell effect sizes}

% Clustering of the effect-size profiles of 322 significant context-specific eQTL (FDR $<$ 5\%) identified using 10 MOFA factors as contexts was performed using SpatialDE, by using the same first 10 MOFA factors as spatial coordinates, and default parameters. 
% Effect sizes were standardized prior to clustering (mean=0, standard deviation=1). 
% This identified 17 clusters, containing between 2 and 66 genes.

% \subsection{Cluster enrichment}

% For each cluster, I considered Pearson's correlation between the cluster's summary profiles and single-cell gene expression across all genes (n=11,231).
% Next, I selected all genes with absolute correlation larger than 0.5 (for X/17 clusters no genes were selected) and used gprofiler \cite{raudvere2019g} to identify enriched pathways. 
% Finally, I selected the top XX significant (adjusted p-values < 0.05) pathways per cluster 
