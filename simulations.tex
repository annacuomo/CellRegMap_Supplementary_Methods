% \section{Application to simulated data}

% % First, we applied the model on simulated data.

% First, I used simulation experiments to show calibration of sc-StructLMM.
% % , and then to demonstrate the 
% % power 
% % advantages of scStructLMM compared to other methods. 
% % forthe interaction and association test (Section 2.3). 
% I use this section to describe the simulation procedure used to generate these simulated data and the methods used to assess calibration.
% % and statistical power. 
% % I will then show calibration and power results for some general settings, followed by results for simulation experiments that explicitly examine the effect of certain environmental properties.


% \subsection{Simulation strategy}
\section{Simulation strategy}\label{sec:simulations}

We used simulation experiments to show calibration and power of CellRegMap.
The code for building simulations generated from the model can be found at \url{https://github.com/annacuomo/CellRegMap_analyses/tree/main/simulations}.

\subsection{Genotype data}

Given $S$ number of SNPs, we first generated $S$ random minor allele frequencies (MAFs) between 0.2 and 0.45.
Given those, and a number of donors $N$, we generated $S$ independent $N \times 1$ $\mathbf{g}$ vectors as random vectors of 0, 1, 2 with probabilities given by the corresponding MAF ($p(2) = MAF^2$, $p(0) = (1-MAF)^2$, and $p(1) = 1 - p(0) - p(2)$). 
These represent the genotype vectors.

% consider using real genotype vectors

\subsection{Modelling number of cells per donor}\label{sec:cells_per_donor}

We considered three settings when modelling the number of cells per donor. 
For both calibration and power simulations, we considered a fixed number of cells per donor, i.e., 50 cells per donor for 50 donors, resulting in a total of 2,500 samples (cells). 
For the calibration experiments, we additionally considered simulations without such repeat structure, simulating only 1 cell for 2,500 donors. 
Lastly, to evaluate the effects of non-uniform numbers of cells on discovery power, we also simulated a variable number of cells. 
More specifically, the number of cells for donor $i\in\{1,\dots,50\}$ was calculated as

\begin{equation}
   \text{round}(2i\cdot 50 / (50 + 1)),
\end{equation}

resulting again in 2,500 cells in total. 
The genotype vectors ($\mathbf{g}$'s) are expanded appropriately to account for the repeat structure (such that all cells from an individual are assigned the same genotype value).


\subsection{Relatedness matrix}

% Next, I defined the realised relatedness (IBD\footnote{identical by descent}) matrix as the $N \times N$ matrix $\mathbf{R} = \mathbf{G}\mathbf{G}^T$, where $\mathbf{G}$ is the $N \times S$ genotype matrix from above ($\mathbf{g}$ vectors as columns).
% % $\mathbf{R}$ was then column-standardised. % ? check

% consider using real kinship
% consider using just block-diagonal matrix
We defined $\mathbf{R}$ as the repeatedness matrix, by building a block-diagonal matrix, where each block represents an individual and its size depends on the number of cells for that donor.

\subsection{Cellular Contexts}

The cell-cell context covariance matrix was defined as $\boldsymbol{\Sigma} = \mathbf{E}\mathbf{E}^T$, where $\mathbf{E}$ corresponded to a subset of principal component (PC) embeddings calculated from scRNA-seq of differentiating iPS cell lines \cite{cuomo2020single}. 
Unless otherwise stated, we used 20 PCs for all experiments. 
When the number of cells per donor was larger than one, we sampled cells from the full embedding matrix such that all cells for one donor shared the same original cell line.

\subsection{Phenotype}

Phenotype vectors were simulated under both Gaussian and negative binomial noise models. 
Briefly, in the Gaussian case we simulated from the full model (Eq.\eqref{eq:scStructLMM}), using only an intercept as covariate, and a single SNP ($\mathbf{g}_G$) having a persistent effect on $\mathbf{y}$ and another ($\mathbf{g}_{GxC}$) having GxC effects, such that:  

\begin{equation}\label{eq:simulated_pheno}
 \mathbf{y} = \mathbf{y}_0 + \underbrace{\mathbf{g}_G\beta_G}_{\mathbf{y}_G} + \underbrace{\mathbf{g}_{GxC} \odot \boldsymbol{\beta_{GxC}}}_{\mathbf{y}_{GxC}} + \underbrace{\mathbf{c}}_{\mathbf{y}_C} + \underbrace{\mathbf{u}}_{\mathbf{y}_{RC}} + \underbrace{\boldsymbol{\psi}}_{\mathbf{y}_n}. 
\end{equation}

% where $\mathbf{e} \sim \mathcal{N} (\mathbf{0}, \sigma_E^2 \mathbf{I})$. 

\subsubsection{Variance terms}\label{sec:variances}

We column-normalised all terms so that the total variance sums to 1 (see details below). Next, we set the variance explained by both genetic terms $\mathrm{var}_G+\mathrm{var}_{GxC}=\sigma_0^2$, and call the rest $v = 1-\sigma_0^2$.
Further, we regulated the amount of variance driven by GxC using an additional weighting factor $\rho_0$, such that: $\mathrm{var}_G = (1-\rho_0)\sigma_0^2$ and $\mathrm{var}_{GxC} = \rho_0\sigma_0^2$.
For simplicity, the variances explained by the last three terms were set to the same value:
$\sigma_C^2 = \sigma_{RC}^2 = \sigma_n^2 = v/3$ for Gaussian and $\sigma_C^2 = \sigma_{RC}^2 = v/2, \sigma_n^2 = 0$ in the case of a negative binomial noise model, respectively, such that:

\begin{equation}\label{eq:variances}
 \mathbf{y} = \mathbf{y}_0 + \underbrace{\underbrace{\mathbf{g}_G\beta_G}_{\mathrm{var}_G} + \underbrace{\mathbf{g}_{GxC} \odot \boldsymbol{\beta_{GxC}}}_{\mathrm{var}_{GxC}}}_{\sigma_0^2} + \underbrace{\underbrace{\mathbf{e}}_{\sigma_C^2} + \underbrace{\mathbf{u}}_{\sigma_{RC}^2} + \underbrace{\boldsymbol{\psi}}_{\sigma_n^2}}_{v}. 
\end{equation}

\subsubsection{Persistent genetic effects}

First, we define an $S \times 1$ effect size vector $\bfbeta$, which represent the effect size of each of the $S$ SNPs modelled. 
We model most SNPs to have no effect $\beta_i=0$ except for one `causal' SNP ($\mathbf{g}_G$), which we randomly set to either $1$ or $-1$ (i.e., $\beta_G=\pm1$).
Next, we scale the vector to ensure that the variance explained by $\mathbf{y}_G$ is $\mathrm{var}_G$:

\begin{equation*}
    \bfbeta_s = \frac{1}{\sqrt{\mathrm{var}_G}} \bfbeta.
\end{equation*}

Finally, we compute the persistent genetic effect component of the phenotype vector by multiplying the genotype matrix $\mathbf{G}$ by the resulting vector scaled $\bfbeta_s$, i.e.,:

\begin{equation}
   \mathbf{y}_G =  \mathbf{G} \bfbeta_s.
\end{equation}

\subsubsection{GxC effects}

% If $N$ is the total number of samples (all cells), and $\mathbf{y}$ is a $N \times 1$ vector.
% We have $\mathbf{E}$ a $N \times n_E$ vector, then $\mathbf{g}_i$ is $N \times 1$, then $\boldsymbol{\alpha}_i$ is $n_E \times 1$

% Let $i$ denote a SNP index and $j$ denote an environment.
For every SNP $i$, let $\mathbf{y}_{G\times C,i}= \mathbf{g}_i \odot (\mathbf{E} \boldsymbol{\alpha_i})$ be its G$\times$C effect, where $\mathbf{E}$ is the $N \times n_E$ cellular environment matrix and $\boldsymbol{\alpha}_i$ is a $n_E \times 1$ normally distributed random variable such that:

\begin{align}
    \boldsymbol{\alpha}_i \sim \mathcal{N}(0,\sigma^2_{G\times C,i}\mathbf{I}_{n_E}).
\end{align}

% We have

% \begin{align}
%     \mathrm{E}[\mathbf{y}_{GxE,i}] = \mathrm{E}[\mathbf{g}_i \odot (\mathbf{E} \boldsymbol{\alpha_i})] =  \mathrm{E}[\mathbf{g}_i] \ \mathrm{E}[\mathbf{E} \boldsymbol{\alpha_i}] = \mathbf{0} \odot \mathrm{E}[\mathbf{E}\boldsymbol{\alpha_i}] = \mathbf{0},
% \end{align}

% where we use that $\mathbf{g}_i$ is standardised ($\mathrm{E}[\mathbf{g}_i]=0$) and we assume that $\mathbf{g}_i$ and $(\mathbf{E}\boldsymbol{\alpha_i})$ are uncorrelated. \\


% We also have (for every SNP $i$, environment $j$):

% \begin{align}
%     \mathrm{E}[\mathbf{y}_{GxE,i}^2] =  \mathrm{E}[(\mathbf{g}_i \odot \mathbf{E} \boldsymbol{\alpha_i})^2] = \sum_j \mathrm{E}[\epsilon_j^2] \ \mathrm{E}[\alpha_{i,j}^2] = \sigma_i^2 \mathbf{I},
% \end{align}

% after a couple of assumptions.
Similarly to before, to ensure that the variance explained is $\mathrm{var}_{GxC}$, we set $\sigma^2_{GxC,i} = \mathrm{var}_{GxC}$
% We define $\sigma_i^2=v_i$ 
if $\mathbf{g}_i$ is causal (assuming once again a single SNP being causal for G$\times$C effects) and $\sigma_{GxC,i}^2=0$ otherwise.
% We assume all causal SNPs to have equal effect as defined by $v_i=\sigma^2/n_{GxE}$, where $n_{GxE}$ is the number of SNPs having GxE effects.
% We also assume that $\mathrm{E}[\epsilon_j]=0$ and $\mathrm{E}[\epsilon_j^2]=\frac{1}{n_E}$ for every environment $j$. \\
The G$\times$C component of the phenotype can then be obtained as:

\begin{equation}
   \mathbf{y}_{GxC} = \sum_i \mathbf{y}_{GxC,i}= \sum_{i} \mathbf{g}_i \odot  \mathbf{E} \boldsymbol{\alpha_i}.
\end{equation}

\subsubsection{Additive random effects}

Next, we simulate the last three random effect terms, noting that we simulate them all to explain the same variance. \\
% Scale variance, build normally distributed vector, multiply by design matrix.

First, the effect of cellular contexts directly:

\begin{equation}
   \mathbf{y}_{C} = \mathbf{E} (\sigma_{C} \ \mathbf{n}), ~~~ \mathrm{with} \ \mathbf{n} \sim \mathcal{N}(\mathbf{0},\mathbf{I}).
\end{equation}

Next, the background term to account for relatedness ($\mathbf{R} \odot \mathbf{E}\mathbf{E}^T$):

\begin{equation}
  \mathbf{y}_{RC} = [\mathbf{Lk}_1 ... \mathbf{Lk}_{n_E}] (\sigma_{RE} \ \mathbf{n}), ~~~ \mathrm{with} \ \mathbf{n} \sim \mathcal{N}(\mathbf{0},\mathbf{I}),
\end{equation}

% \begin{equation}
%   \mathbf{y}_{RE} = \{\mathbf{hRE} (\sqrt{v/3} \ \mathbf{n}), ~~~ \mathrm{with} \ \mathbf{n} \sim \mathcal{N}(\mathbf{0},\mathbf{I}),
% \end{equation}

% where $\mathbf{hRE} \ \mathbf{hRE}^T = \mathbf{R} \odot \mathbf{E}\mathbf{E}^T$.

where $\mathbf{Lk}$'s are as in Eq. \eqref{eq:Lk}.

\subsubsection{Noise}\label{sec:noise}

For Gaussian noise we model the noise term as:

\begin{equation}
   \mathbf{y}_{n} = \sigma_n \ \mathbf{n}, ~~~ \mathrm{with} \ \mathbf{n} \sim \mathcal{N}(\mathbf{0},\mathbf{I}).
\end{equation}

Alternatively, we consider a negative binomial noise model, where we set $\boldsymbol{\psi}=\mathbf{y}_n=\mathbf{0}$ and map $\mathbf{y}$ to the mean of a negative binomial distribution using a $\log$ link function. The probability mass function of a negative binomial is commonly parameterized as
\begin{equation}
    P(k; r,p)=\frac{\Gamma(k+r)}{k!\Gamma(r)}p^r(1-p)^k
\end{equation}
where $k+r$ is the number of trials and $r$ is the number of successes for a probability of success $p$. We parameterize the distribution in terms of its mean $\mu$ and dispersion parameter $\phi$, such that
\begin{align}
    r&=\phi^{-1} \\
    p&=\frac{r}{r+\mu}
\end{align}

The dispersion parameter of the negative binomial was set to $\phi=1.5$ for all simulations, and the intercept was set to $y_0=2.5$.

% \newpage

\section{Model validation using simulated data}
\label{sec:validation}

In order to validate the calibration of CellRegMap and to assess statistical power of the model, we considered simulated data generated as described in Section~\ref{sec:simulations}.
For comparison, we also considered two alternative models as described in the next section.

\subsection{Alternative methods}
We considered different comparison methods for calibration and power assessment of our model.

\subsubsection{StructLMM}

First, we assessed calibration of our method in comparison to the original StructLMM model, thus comparing StructLMM:

\begin{equation}
 \mathbf{y} = \mathbf{g}\beta_G + \mathbf{g} \odot \boldsymbol{\beta_{GxC}} + \mathbf{c} +\boldsymbol{\psi}
\end{equation}

vs CellRegMap:

\begin{equation}
 \mathbf{y} = \mathbf{g}\beta_G + \mathbf{g} \odot \boldsymbol{\beta_{GxC}} + \mathbf{c} + \mathbf{u} +\boldsymbol{\psi},
\end{equation}

where $\mathbf{y}$, $\mathbf{g}$, $\beta_G$, $\mathbf{g}$, $\boldsymbol{\beta_{GxC}}$, $\mathbf{c}$, $\mathbf{u}$ and $\boldsymbol{\psi}$ are as previously described (Eqs. \eqref{eq:StructLMM}, \eqref{eq:scStructLMM}). 
This comparison was conducted to confirm the need for the relatedness component (using a random effect component $\mathbf{u}$).

\subsubsection{SingleEnv-Int}

For power analyses, we compared CellRegMap  to one more traditionally used for interaction tests (e.g., by \cite{zhernakova2017identification, van2020single1}), which consists of including in the model an interaction term to account for interactions between genotype and one environmental covariate at a time:

\begin{equation}
 \mathbf{y} = \mathbf{g}\beta_G + \mathbf{E_i}\gamma + \mathbf{g} \mathbf{E_i} \beta_{GxC} + \mathbf{u} +\boldsymbol{\psi}.
\end{equation}

Here, $\mathbf{E_i}$ denotes a single cellular environment, i.e., one column from matrix $\mathbf{E}$. 
The parameter $\gamma$ denotes the additive effect of the environment and $\beta_{GxC}$ denotes the effect size of the interaction term.
To facilitate a direct comparison, this model employs the same relatedness component as used in CellRegMap ($\mathbf{u}$).
Note that using this approach involves performing as many tests as there are cellular contexts ($i = 1,..,n_E$), and thus needs to be followed by multiple testing correction.
The results report in Fig. 2 are based on Bonferroni adjustment for the number of environmental variables considered.

\subsection{Assessment of statistical calibration and power}

We assessed calibration of both CellRegMap and StructLMM under null, assuming no genetic effects (i.e. $\sigma_0^2 = 0$, section \ref{sec:variances}) as well as in the case of a persistent genetic effects only (i.e. $\text{var}_{GxC} = 0$, see section \ref{sec:variances}).
We repeat this simulation study either simulating  Gaussian or negative binomial distributed observations (Supplementary Fig. 2.1).
In all cases, 200 SNPs were tested and the resulting p-values were compared to values drawn from a uniform distribution to assess calibration (QQ-plots).
All analyses were performed both in the case of no relatedness, modelling 2,500 independent cells (one cell per individual) and in the presence of repeated structure (50 cells for each of 50 individuals, resulting in 2,500 cells in total). \\

For our power analysis, we compared CellRegMap to the SingleEnv-Int model described above, followed by Bonferroni correction across the contexts tested.
Power was assessed as the true positive rate across 250 simulated genes.
We assessed power between our model and the conventional single-environment interaction test as a function of i) the fraction of genetic variance explained by G$\times$C ($\rho_0$), ii) the number of simulated contexts with G$\times$C and iii) the number of tested cellular environments.
We simulated both a fixed number of cells per donor (50 cells per donor, as above), as well as variable number of cells per individual (see section \ref{sec:cells_per_donor}). \\

Both for the calibration and power analysis, we either considered the setting of directly simulating Gaussian distributed phenotypes, or alternatively simulating RNA-seq counts by drawing from a negative binomial model (see section \ref{sec:noise}); see Fig. 2 and Supp Fig. 2.1 \& 2.2. In the case of simulating counts, we used the same preprocessing as employed on real data (quantile normalization of $\log$-transformed counts) to define the input phenotype for CellRegMap and alternative methods. 


