\section{CellRegMap-Association test} \label{sec:assoc}

Within the CellRegMap software suite it is also possible to test for persistent genetic effects, while appropriately account for cellular context.
This model essentially draws from the null hypothesis of CellRegMap, thus comparing the following two models (where we exclude the G$\times$ term from Eq. \eqref{eq:scStructLMM}):

\begin{equation*}
    H_0: \mathbf{y} = \mathbf{W}\boldsymbol{\alpha} + \mathbf{c} + \mathbf{u} + \boldsymbol{\psi},
\end{equation*}

vs

\begin{equation*}
    H_1: \mathbf{y} = \mathbf{W}\boldsymbol{\alpha} + \mathbf{g}\beta_G + \mathbf{c} + \mathbf{u} + \boldsymbol{\psi}.
\end{equation*}

All variables are defined as in section~\ref{sec:CellRegMap}, with the exception of $\mathbf{u}$, which here is defined to account for the repeated structure due to many cell been drawn from the same individual:

\begin{equation*}
    \mathbf{u} \sim \mathcal{N}(\mathbf{0},\sigma_{r}^2 \mathbf{R}).
\end{equation*}

This simplifies the implementation, where we note that Eq. \eqref{K0} becomes:

\begin{equation}\label{K0}
\begin{split}
    \mathbf{K}_0 = \sigma_c^2 \boldsymbol{\Sigma} + \sigma_{r}^2 \mathbf{R} + \sigma_n^2 \mathbf{I} =\\
    \sigma_m^2 \ [\rho_1\boldsymbol{\Sigma} + (1-\rho_1) \mathbf{R} ] + \sigma_n^2 \mathbf{I} =\\ \sigma_m^2 \{ {\mathbf{M}'(\rho_1) + \delta_1 \mathbf{I}\}},
\end{split}
\end{equation}

from which follows:

\begin{equation}
    \mathbf{M} = \boldsymbol{\Sigma} + \mathbf{R},  
\end{equation}

and

\begin{equation}
    \mathbf{N} = [\mathbf{E} \ | \mathbf{G}],
\end{equation}

where we assume, as before, $\boldsymbol{\Sigma} = \mathbf{E}\mathbf{E}^T$ and $\mathbf{R} = \mathbf{G}\mathbf{G}^T$.

To assess significance, we use a likelihood ratio test (LRT).
This is well defined since we are testing $\beta_G \neq 0$, which is not at the boundary of possible values of the parameter, as $-\infty < \beta_G < \infty$.

